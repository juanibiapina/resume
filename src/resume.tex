%%%%%%%%%%%%%%%%%
% This is an sample CV template created using altacv.cls
% (v1.7.2, 28 August 2024) written by LianTze Lim (liantze@gmail.com). Compiles with pdfLaTeX, XeLaTeX and LuaLaTeX.
%
%% It may be distributed and/or modified under the
%% conditions of the LaTeX Project Public License, either version 1.3
%% of this license or (at your option) any later version.
%% The latest version of this license is in
%%    http://www.latex-project.org/lppl.txt
%% and version 1.3 or later is part of all distributions of LaTeX
%% version 2003/12/01 or later.
%%%%%%%%%%%%%%%%

%% Use the "normalphoto" option if you want a normal photo instead of cropped to a circle
% \documentclass[10pt,a4paper,normalphoto]{altacv}

\documentclass[10pt,a4paper,ragged2e,withhyper]{altacv}
%% AltaCV uses the fontawesome5 and simpleicons packages.
%% See http://texdoc.net/pkg/fontawesome5 and http://texdoc.net/pkg/simpleicons for full list of symbols.

% Change the page layout if you need to
\geometry{left=1.25cm,right=1.25cm,top=1.5cm,bottom=1.5cm,columnsep=1.2cm}

% The paracol package lets you typeset columns of text in parallel
\usepackage{paracol}

% Improve microtypography (justification, mainly)
\usepackage{microtype}

% Change the font if you want to, depending on whether
% you're using pdflatex or xelatex/lualatex
% WHEN COMPILING WITH XELATEX PLEASE USE
% xelatex -shell-escape -output-driver="xdvipdfmx -z 0" sample.tex
\iftutex
  % If using xelatex or lualatex:
  \setmainfont{Roboto Slab}
  \setsansfont{Lato}
  \renewcommand{\familydefault}{\sfdefault}
\else
  % If using pdflatex:
  \usepackage[rm]{roboto}
  \usepackage[defaultsans]{lato}
  % \usepackage{sourcesanspro}
  \renewcommand{\familydefault}{\sfdefault}
\fi

% Change the colours if you want to
\definecolor{SlateGrey}{HTML}{2E2E2E}
\definecolor{LightGrey}{HTML}{666666}
\definecolor{DarkPastelRed}{HTML}{450808}
\definecolor{PastelRed}{HTML}{8F0D0D}
\definecolor{GoldenEarth}{HTML}{E7D192}
\colorlet{name}{black}
\colorlet{tagline}{PastelRed}
\colorlet{heading}{DarkPastelRed}
\colorlet{headingrule}{GoldenEarth}
\colorlet{subheading}{PastelRed}
\colorlet{accent}{PastelRed}
\colorlet{emphasis}{SlateGrey}
\colorlet{body}{LightGrey}

% Change some fonts, if necessary
\renewcommand{\namefont}{\Huge\rmfamily\bfseries}
\renewcommand{\personalinfofont}{\footnotesize}
\renewcommand{\cvsectionfont}{\LARGE\rmfamily\bfseries}
\renewcommand{\cvsubsectionfont}{\large\bfseries}


% Change the bullets for itemize and rating marker
% for \cvskill if you want to
\renewcommand{\cvItemMarker}{{\small\textbullet}}
\renewcommand{\cvRatingMarker}{\faCircle}
% ...and the markers for the date/location for \cvevent
% \renewcommand{\cvDateMarker}{\faCalendar*[regular]}
% \renewcommand{\cvLocationMarker}{\faMapMarker*}

% new command for accent color
\newcommand{\accentbold}[1]{\textbf{\textcolor{accent}{#1}}}

% If your CV/résumé is in a language other than English,
% then you probably want to change these so that when you
% copy-paste from the PDF or run pdftotext, the location
% and date marker icons for \cvevent will paste as correct
% translations. For example Spanish:
% \renewcommand{\locationname}{Ubicación}
% \renewcommand{\datename}{Fecha}


%% Use (and optionally edit if necessary) this .tex if you
%% want to use an author-year reference style like APA(6)
%% for your publication list
% \input{pubs-authoryear.cfg}

%% Use (and optionally edit if necessary) this .tex if you
%% want an originally numerical reference style like IEEE
%% for your publication list
\input{pubs-num.cfg}

%% sample.bib contains your publications
%\addbibresource{sample.bib}

% Disable hyphenation
\usepackage[none]{hyphenat}

\begin{document}
\name{Juan Ibiapina}
\tagline{Software Engineer}
%% You can add multiple photos on the left or right
% \photoR{2.8cm}{Globe_High} % TODO: Set my photo
% \photoL{2.5cm}{Yacht_High,Suitcase_High}

\personalinfo{%
  % Not all of these are required!
  \email{juanibiapina@gmail.com}
  %\phone{000-00-0000}
  %\mailaddress{Åddrésş, Street, 00000 Cóuntry}
  \location{Berlin, Germany}
  \homepage{juanibiapina.github.io}
  % \twitter{@twitterhandle}
  %\xtwitter{@x-handle}
  \linkedin{juan-i-10025313}
  \github{juanibiapina}
  %\orcid{0000-0000-0000-0000}
  %% You can add your own arbitrary detail with
  %% \printinfo{symbol}{detail}[optional hyperlink prefix]
  % \printinfo{\faPaw}{Hey ho!}[https://example.com/]

  %% Or you can declare your own field with
  %% \NewInfoFiled{fieldname}{symbol}[optional hyperlink prefix] and use it:
  % \NewInfoField{gitlab}{\faGitlab}[https://gitlab.com/]
  % \gitlab{your_id}
  %%
  %% For services and platforms like Mastodon where there isn't a
  %% straightforward relation between the user ID/nickname and the hyperlink,
  %% you can use \printinfo directly e.g.
  % \printinfo{\faMastodon}{@username@instace}[https://instance.url/@username]
  %% But if you absolutely want to create new dedicated info fields for
  %% such platforms, then use \NewInfoField* with a star:
  % \NewInfoField*{mastodon}{\faMastodon}
  %% then you can use \mastodon, with TWO arguments where the 2nd argument is
  %% the full hyperlink.
  % \mastodon{@username@instance}{https://instance.url/@username}
}

\makecvheader
%% Depending on your tastes, you may want to make fonts of itemize environments slightly smaller
% \AtBeginEnvironment{itemize}{\small}

%% Set the left/right column width ratio to 6:4.
\columnratio{0.6}

% Start a 2-column paracol. Both the left and right columns will automatically
% break across pages if things get too long.
\begin{paracol}{2}

\cvsection{Experience}

\cvevent{Staff Software Engineer}{Babbel}{Aug 2017 -- Feb 2025}{Berlin, Germany}

{\RaggedRight
I worked in the Content Platform team, developing the learning content CMS and related content distribution \accentbold{APIs}. We designed \accentbold{cross-team architecture}, striving for \accentbold{small deliverables} that bring value to users and allow teams to achieve their \accentbold{goals} and work \accentbold{independently}.

\medskip

\begin{itemize}
\item \accentbold{Content Pipeline}: Designed an integration between Babbel's \accentbold{CMS} and \accentbold{Contentful}, enabling fast creation of content for new learning experiences. This system now holds content for more than 20 learning experiences, fulfilling the company goal of experimenting with new forms of learning. Teams can \accentbold{autonomously} create new types through Pull Requests. All content is \accentbold{Git} versioned and \accentbold{immutable}.
\item \accentbold{Content Deployment}: Led a multi-year project to enable content editors to deploy content increasing deployments from one per quarter to several per week.
\item \accentbold{Content Modeling}: Designed an extensible new data model for current and future Babbel content, inspired by \accentbold{NixOS} and served through a \accentbold{GraphQL} API. The new model is backwards compatible but also enables new use cases involving personalization and AI integration.
\item \accentbold{Content APIs}: Designed \accentbold{RESTful} content APIs for current and future company initiatives, focusing on \accentbold{performance} and \accentbold{cacheability}.
\item \accentbold{Workshops}: Organized internal \accentbold{presentations} and \accentbold{workshops} about Babbel's content domain and architecture, \accentbold{empowering} other teams and content creators.
\item \accentbold{API Gateway tooling}: Improved build times (\accentbold{from 15m to 5m}) and tooling for Babbel's main API Gateway, shared between teams. Engineers can simply run `make` and all checks run locally with zero setup required. The same setup runs on \accentbold{CI}.
\item \accentbold{Engineer-friendly Documentation}: Started a documentation initiative to create engineer-friendly documentation in repositories. Many teams have adopted this practice and maintain Git versioned documentation.
\item \accentbold{User vocabulary migration}: Migrated all user vocabulary from \accentbold{MySql} to \accentbold{DynamoDB}. It went from 7 unmaintainable, untested joins, to a performant and scalable single table design.
\item \accentbold{Service extraction}: Actively drove or participated in many service extractions from the monolith, including user vocabulary, content, authorization, b2b, accounts and user progress \accentbold{services}.
\item \accentbold{API Tests}: Wrote a suite of \accentbold{API Integration} tests (mostly for myself, at the beginning). It has grown and is now a valuable tool used by many teams to find regressions.
\item \accentbold{Mentoring}: Mentored \accentbold{junior}, \accentbold{professional} and \accentbold{senior} engineers, directly supporting their \accentbold{growth} to higher roles, including several promotions to \accentbold{Staff}.
\item \accentbold{Advice and support}: Teams contact me on a weekly basis to discuss system design. I advised about sound \accentbold{domain entities} with clear nomenclature, well defined \accentbold{system boundaries} and responsibilities, \accentbold{small deliverables}, and then small deliverables a few more times because the \accentbold{project only starts once it's in production}.
\item \accentbold{Staff Role}: I was the first engineer to be promoted to Staff. I helped define the role and its responsibilities drawing from my own experience working across teams.
\end{itemize}
}

\divider

\cvevent{Senior Full-Stack Software Engineer}{Movinga}{Feb 2017 -- July 2017}{Berlin, Germany}

{\RaggedRight
Optimizing pricing and inventory for a moving startup.

\medskip

\begin{itemize}
\item Extracted a pricing microservice from the monolith in order to improve deployment times and frequency (from once every two weeks to 10 times a day).
\item Provided dashboards and metrics about pricing, enabling teams to make data driven decisions.
\end{itemize}
}

\divider

\cvevent{Senior Software Engineer}{Globo.com}{May 2015 -- Jan 2017}{Brazil}

{\RaggedRight
Maintaining a distributed system across more than 50 physical locations for encoding raw television videos for online streaming.

\medskip

\begin{itemize}
\item Designed a content protection microservice that integrated with 3 third party APIs
\item Started an observability initiative to gather video encoding metrics and better understand our bottlenecks.
\item Defined and implemented video formats for streaming 4K videos.
\item Interviewed most people hired in the Porto Alegre branch.
\item Mentored less experienced engineers
\item Helped shape the office culture.
\end{itemize}
}

\divider

\cvevent{Senior Software Engineer}{Bearch, Inc.}{Nov 2014 -- Apr 2015}{USA}

{\RaggedRight
Building an anonymous social network.

\medskip

\begin{itemize}
\item Worked daily with new technologies, including Go, Android, iOS, React and Angular
\item Shipped a huge amount of features in very little time due to our experimental nature
\item Created a pipeline for building and deploying android and iOS apps to their respective stores, automating the testing and release cycles
\item Created a system to deploy code on new environments on Google App Engine, speeding up testing of new features
\item Developed a camera for Android that worked on the new, unconventional screen sizes being released at the time
\end{itemize}
}

\divider

\cvevent{Software Engineer}{e-Core}{Mar 2013 -- Nov 2014}{Brazil}

{\RaggedRight
Developed a Single Page Application for analyzing merger agreements.

\medskip

\begin{itemize}
\item Automated the deployment process and reduced the total time from 2 hours down to 160 seconds
\item Migrated a legacy application from Grails to Rails
\end{itemize}
}

\divider

\cvevent{Software Engineer}{Codeminer42}{Jul 2012 -- Feb 2013}{Brazil}

{\RaggedRight
Consulting and developing for startups.

\medskip

\begin{itemize}
\item Migrated an application from Rails 2 to Rails 4.
\item Developed a project from conception to delivery, including requirements gathering, system design, implementation, infrastructure and monitoring.
\item I was one of the first people to be hired on our branch of the company, so I had the opportunity to help build the culture we wanted. I learned a lot from the experience and it helped shape some of my future career goals.
\end{itemize}
}

\divider

\cvevent{Development Consultant}{ThoughtWorks}{Jan 2011 -- Jun 2012}{Brazil}

{\RaggedRight
Consulting and web development for the retail industry, with daily client facing situations. I worked mostly on maintaining and developing new features for a large legacy Java code base. Teams were distributed across Brazil, India and the US.

\medskip

This was my first contact with agile practices. I have learned much from it, including the importance of pair programming, TDD and Continuous Integration. I had a mentor that helped me build most of my ideas about development.

\medskip

\begin{itemize}
\item Participated on ThoughtWorks University, which changed my life.
\item Created a dashboard to monitor build status and put it on a TV in the office, increasing visibility and reducing the time to fix broken builds.
\item Improved performance of the development environment, greatly improving development speed and team motivation.
\item Acquired the trust of the main decision maker in the client, making our work much easier.
\item Participated on the hiring process for new developers.
\end{itemize}
}

% use ONLY \newpage if you want to force a page break for
% ONLY the current column
%\newpage

%\cvsection{Publications}

%% Specify your last name(s) and first name(s) as given in the .bib to automatically bold your own name in the publications list.
%% One caveat: You need to write \bibnamedelima where there's a space in your name for this to work properly; or write \bibnamedelimi if you use initials in the .bib
%% You can specify multiple names, especially if you have changed your name or if you need to highlight multiple authors.
%\mynames{Lim/Lian\bibnamedelima Tze,
%  Wong/Lian\bibnamedelima Tze,
%  Lim/Tracy,
%  Lim/L.\bibnamedelimi T.}
%% MAKE SURE THERE IS NO SPACE AFTER THE FINAL NAME IN YOUR \mynames LIST

%\nocite{*}

%\printbibliography[heading=pubtype,title={\printinfo{\faBook}{Books}},type=book]

%\divider

%\printbibliography[heading=pubtype,title={\printinfo{\faFile*[regular]}{Journal Articles}},type=article]

%\divider

%\printbibliography[heading=pubtype,title={\printinfo{\faUsers}{Conference Proceedings}},type=inproceedings]

%% Switch to the right column. This will now automatically move to the second
%% page if the content is too long.
\switchcolumn

%\cvsection{My Life Philosophy}
%
%\begin{quote}
%``Something smart or heartfelt, preferably in one sentence.''
%\end{quote}

%\cvsection{Most Proud of}
%
%\cvachievement{\faTrophy}{Fantastic Achievement}{and some details about it}
%
%\divider
%
%\cvachievement{\faHeartbeat}{Another achievement}{more details about it of course}

\cvsection{About Me}

Versatile engineer with \accentbold{14 years of experience}, focused on \accentbold{Backend} but also working with DevOps and Frontend. Strong advocate for \accentbold{code quality}, \accentbold{readability}, \accentbold{TDD} and \accentbold{continuous delivery}. Experienced in designing \accentbold{distributed services}, \accentbold{microservices} and \accentbold{APIs}.

\medskip

I'm experienced in working with cross-functional teams across multiple projects. I thrive in environments where I can solve complex problems, shape architecture and drive initiatives. I collaborate closely with teams and stakeholders to ensure alignment and promote technical excellence.

\cvsection{Skills}

\cvevent{Programming Languages}{Go, Ruby, JavaScript, TypeScript, Bash, Rust}{}{}
\cvevent{Frameworks}{Rails, React, GraphQL, Lambda}{}{}
\cvevent{Tools}{Docker, Git, Nix/NixOS, Neovim, ffmpeg}{}{}
\cvevent{Databases}{Postgres, DynamoDB, Redis, MySQL, MongoDB, Elastic Search}{}{}
\cvevent{DevOps}{CI/CD, Terraform, Rollbar, Datadog, Sentry, CloudWatch}{}{}
\cvevent{Practices}{TDD, Pair Programming, Refactoring}{}{}
\cvevent{Previously Worked With}{Angular, Grape, Grails, Backbone.js, Knockout.js, Puppet, Chef, Solr, VirtualBox, nginx}{}{}

\cvsection{Education}

\cvevent{B.Sc.\ in Computer Science}{UFPI - Universidade Federal do Piauí}{Jan 2004 -- Dec 2007}{}


\cvsection{Languages}

\cvskill{English}{5}
\cvskill{Portuguese}{5}
\cvskill{German}{2} %% Supports X.5 values.


\newpage

\cvsection{Open Source}

\cvevent{\href{https://github.com/basherpm/basher}{basher}}{A package manager for shell scripts}{}{}
\cvevent{\href{https://github.com/juanibiapina/sub}{sub}}{Shell scripts with superpowers}{}{}
\cvevent{\href{https://github.com/juanibiapina/dotfiles}{dotfiles}}{My dotfiles for NixOS and Neovim}{}{}

%\cvsection{Referees}
%
%% \cvref{name}{email}{mailing address}
%\cvref{Prof.\ Alpha Beta}{Institute}{a.beta@university.edu}
%{Address Line 1\\Address line 2}
%
%\divider
%
%\cvref{Prof.\ Gamma Delta}{Institute}{g.delta@university.edu}
%{Address Line 1\\Address line 2}


\end{paracol}


\end{document}
