%%%%%%%%%%%%%%%%%
% This is an sample CV template created using altacv.cls
% (v1.7.2, 28 August 2024) written by LianTze Lim (liantze@gmail.com). Compiles with pdfLaTeX, XeLaTeX and LuaLaTeX.
%
%% It may be distributed and/or modified under the
%% conditions of the LaTeX Project Public License, either version 1.3
%% of this license or (at your option) any later version.
%% The latest version of this license is in
%%    http://www.latex-project.org/lppl.txt
%% and version 1.3 or later is part of all distributions of LaTeX
%% version 2003/12/01 or later.
%%%%%%%%%%%%%%%%

%% Use the "normalphoto" option if you want a normal photo instead of cropped to a circle
% \documentclass[10pt,a4paper,normalphoto]{altacv}

\documentclass[10pt,a4paper,ragged2e,withhyper]{altacv}
%% AltaCV uses the fontawesome5 and simpleicons packages.
%% See http://texdoc.net/pkg/fontawesome5 and http://texdoc.net/pkg/simpleicons for full list of symbols.

% Change the page layout if you need to
\geometry{left=1.25cm,right=1.25cm,top=1.5cm,bottom=1.5cm,columnsep=1.2cm}

% The paracol package lets you typeset columns of text in parallel
\usepackage{paracol}

% Improve microtypography (justification, mainly)
\usepackage{microtype}

% Change the font if you want to, depending on whether
% you're using pdflatex or xelatex/lualatex
% WHEN COMPILING WITH XELATEX PLEASE USE
% xelatex -shell-escape -output-driver="xdvipdfmx -z 0" sample.tex
\iftutex
  % If using xelatex or lualatex:
  \setmainfont{Roboto Slab}
  \setsansfont{Lato}
  \renewcommand{\familydefault}{\sfdefault}
\else
  % If using pdflatex:
  \usepackage[rm]{roboto}
  \usepackage[defaultsans]{lato}
  % \usepackage{sourcesanspro}
  \renewcommand{\familydefault}{\sfdefault}
\fi

% Change the colours if you want to
\definecolor{SlateGrey}{HTML}{2E2E2E}
\definecolor{LightGrey}{HTML}{666666}
\definecolor{DarkPastelRed}{HTML}{450808}
\definecolor{PastelRed}{HTML}{8F0D0D}
\definecolor{GoldenEarth}{HTML}{E7D192}
\colorlet{name}{black}
\colorlet{tagline}{PastelRed}
\colorlet{heading}{DarkPastelRed}
\colorlet{headingrule}{GoldenEarth}
\colorlet{subheading}{PastelRed}
\colorlet{accent}{PastelRed}
\colorlet{emphasis}{SlateGrey}
\colorlet{body}{LightGrey}

% Change some fonts, if necessary
\renewcommand{\namefont}{\Huge\rmfamily\bfseries}
\renewcommand{\personalinfofont}{\footnotesize}
\renewcommand{\cvsectionfont}{\LARGE\rmfamily\bfseries}
\renewcommand{\cvsubsectionfont}{\large\bfseries}


% Change the bullets for itemize and rating marker
% for \cvskill if you want to
\renewcommand{\cvItemMarker}{{\small\textbullet}}
\renewcommand{\cvRatingMarker}{\faCircle}
% ...and the markers for the date/location for \cvevent
% \renewcommand{\cvDateMarker}{\faCalendar*[regular]}
% \renewcommand{\cvLocationMarker}{\faMapMarker*}


% If your CV/résumé is in a language other than English,
% then you probably want to change these so that when you
% copy-paste from the PDF or run pdftotext, the location
% and date marker icons for \cvevent will paste as correct
% translations. For example Spanish:
% \renewcommand{\locationname}{Ubicación}
% \renewcommand{\datename}{Fecha}


%% Use (and optionally edit if necessary) this .tex if you
%% want to use an author-year reference style like APA(6)
%% for your publication list
% \input{pubs-authoryear.cfg}

%% Use (and optionally edit if necessary) this .tex if you
%% want an originally numerical reference style like IEEE
%% for your publication list
\input{pubs-num.cfg}

%% sample.bib contains your publications
%\addbibresource{sample.bib}

% Disable hyphenation
\usepackage[none]{hyphenat}

\begin{document}
\name{Juan Ibiapina}
\tagline{Software Engineer}
%% You can add multiple photos on the left or right
% \photoR{2.8cm}{Globe_High} % TODO: Set my photo
% \photoL{2.5cm}{Yacht_High,Suitcase_High}

\personalinfo{%
  % Not all of these are required!
  \email{juanibiapina@gmail.com}
  %\phone{000-00-0000}
  %\mailaddress{Åddrésş, Street, 00000 Cóuntry}
  \location{Berlin, Germany}
  %\homepage{www.homepage.com}
  % \twitter{@twitterhandle}
  %\xtwitter{@x-handle}
  \linkedin{juan-i-10025313}
  \github{juanibiapina}
  %\orcid{0000-0000-0000-0000}
  %% You can add your own arbitrary detail with
  %% \printinfo{symbol}{detail}[optional hyperlink prefix]
  % \printinfo{\faPaw}{Hey ho!}[https://example.com/]

  %% Or you can declare your own field with
  %% \NewInfoFiled{fieldname}{symbol}[optional hyperlink prefix] and use it:
  % \NewInfoField{gitlab}{\faGitlab}[https://gitlab.com/]
  % \gitlab{your_id}
  %%
  %% For services and platforms like Mastodon where there isn't a
  %% straightforward relation between the user ID/nickname and the hyperlink,
  %% you can use \printinfo directly e.g.
  % \printinfo{\faMastodon}{@username@instace}[https://instance.url/@username]
  %% But if you absolutely want to create new dedicated info fields for
  %% such platforms, then use \NewInfoField* with a star:
  % \NewInfoField*{mastodon}{\faMastodon}
  %% then you can use \mastodon, with TWO arguments where the 2nd argument is
  %% the full hyperlink.
  % \mastodon{@username@instance}{https://instance.url/@username}
}

\makecvheader
%% Depending on your tastes, you may want to make fonts of itemize environments slightly smaller
% \AtBeginEnvironment{itemize}{\small}

%% Set the left/right column width ratio to 6:4.
\columnratio{0.6}

% Start a 2-column paracol. Both the left and right columns will automatically
% break across pages if things get too long.
\begin{paracol}{2}

\cvsection{Experience}

\cvevent{Staff Software Engineer}{Babbel}{Aug 2017 -- Feb 2025}{Berlin, Germany}

{\RaggedRight
As a Principal at Babbel, I work in the Content Platform team, developing the learning content CMS and related content distribution \textbf{APIs}. We design \textbf{cross-team architecture}, striving for \textbf{small deliverables} that bring value to users and allow teams to achieve their \textbf{goals} and work \textbf{independently}. This is mostly \textbf{remote work}, but we like to meet once per week. Accomplishments:

\begin{itemize}
\item \textbf{Content Pipeline}: Designed an integration between Babbel's \textbf{CMS} and \textbf{Contentful}, enabling fast creation of content for new learning experiences. This system now holds content for more than 20 learning experiences, fulfilling the company goal of experimenting with new forms of learning. Teams can \textbf{autonomously} create new types through Pull Requests. All content is \textbf{Git} versioned and \textbf{immutable}.
\item \textbf{Content Deployment}: Led a multi-year project to enable content editors to deploy content. Through a series of \textbf{modeling} changes, API improvements, \textbf{monitoring}, data migrations, \textbf{test} improvements, education and even a Slack bot, Babbel went from one painful content deployment per quarter to several uneventful content deployments per week.
\item \textbf{Content Modeling}: Designed an extensible new data model for current and future Babbel content, inspired by \textbf{NixOS} and served through a \textbf{GraphQL} API. The new model is backwards compatible but also enables new use cases involving personalization and AI integration.
\item \textbf{Content APIs}: Designed \textbf{RESTful} content APIs for current and future company initiatives, focusing on \textbf{performance} and \textbf{cacheability}.
\item \textbf{Workshops}: Organized internal \textbf{presentations} and \textbf{workshops} about Babbel's content domain and architecture, \textbf{empowering} other teams and content creators.
\item \textbf{API Gateway tooling}: Improved build times (\textbf{from 15m to 5m}) and tooling for Babbel's main API Gateway, shared between teams. Engineers can simply run `make` and all checks run locally using \textbf{Docker} with zero setup required. The same setup runs on \textbf{CI}.
\item \textbf{Engineer-friendly Documentation}: Started a documentation initiative to create engineer-friendly documentation in repositories. Many teams have adopted this practice and maintain Git versioned documentation.
\item \textbf{User vocabulary migration}: Migrated all user vocabulary from \textbf{MySql} to \textbf{DynamoDB}. It went from 7 unmaintainable, untested joins, to a performant and scalable single table design.
\item \textbf{Service extraction}: Actively drove or participated in many service extractions from the \textbf{Rails} monolith, including user vocabulary, content, authorization, b2b, accounts and user progress \textbf{services}.
\item \textbf{API Tests}: Wrote a suite of \textbf{API Integration} tests (mostly for myself, at the beginning). It has grown and is now a valuable tool used by many teams to find regressions.
\item \textbf{Mentoring}: Mentored \textbf{junior}, \textbf{professional} and \textbf{senior} engineers, directly supporting their \textbf{growth} to higher roles, including several promotions to \textbf{Principal}.
\item \textbf{Advice and support}: Teams contact me on a weekly basis to discuss system design. Babbel already includes \textbf{observability} and \textbf{automated infrastructure} by default, so I often advise about sound \textbf{domain entities} with clear nomenclature, well defined \textbf{system boundaries} and responsibilities, \textbf{small deliverables}, and then small deliverables a few more times because the \textbf{project only starts once it's in production}.
\item \textbf{Principal Role}: I was the first engineer to be promoted to Principal. I helped define the role and its responsibilities drawing from my own experience temporarily moving between teams.
\end{itemize}
}

\divider

\cvevent{Senior Full-Stack Software Engineer}{Movinga}{Feb 2017 -- Aug 2017}{Berlin, Germany}

% use ONLY \newpage if you want to force a page break for
% ONLY the current column
%\newpage

%\cvsection{Publications}

%% Specify your last name(s) and first name(s) as given in the .bib to automatically bold your own name in the publications list.
%% One caveat: You need to write \bibnamedelima where there's a space in your name for this to work properly; or write \bibnamedelimi if you use initials in the .bib
%% You can specify multiple names, especially if you have changed your name or if you need to highlight multiple authors.
%\mynames{Lim/Lian\bibnamedelima Tze,
%  Wong/Lian\bibnamedelima Tze,
%  Lim/Tracy,
%  Lim/L.\bibnamedelimi T.}
%% MAKE SURE THERE IS NO SPACE AFTER THE FINAL NAME IN YOUR \mynames LIST

%\nocite{*}

%\printbibliography[heading=pubtype,title={\printinfo{\faBook}{Books}},type=book]

%\divider

%\printbibliography[heading=pubtype,title={\printinfo{\faFile*[regular]}{Journal Articles}},type=article]

%\divider

%\printbibliography[heading=pubtype,title={\printinfo{\faUsers}{Conference Proceedings}},type=inproceedings]

%% Switch to the right column. This will now automatically move to the second
%% page if the content is too long.
\switchcolumn

%\cvsection{My Life Philosophy}
%
%\begin{quote}
%``Something smart or heartfelt, preferably in one sentence.''
%\end{quote}

%\cvsection{Most Proud of}
%
%\cvachievement{\faTrophy}{Fantastic Achievement}{and some details about it}
%
%\divider
%
%\cvachievement{\faHeartbeat}{Another achievement}{more details about it of course}

\cvsection{About Me}

I have \textbf{14 years of experience} developing software professionally. I focus on \textbf{backend}, but have also done \textbf{frontend}, mobile, and infrastructure. I advocate \textbf{code quality} and \textbf{continuous delivery}. \textbf{Tests} are part of my development practices, along with \textbf{TDD} and refactoring. I’m confident working with \textbf{agile} practices that enable \textbf{fast delivery} in small increments. I also have experience designing distributed services, microservices, and \textbf{REST APIs}. I'm a big fan of \textbf{tooling} and \textbf{observability}.

\begingroup
\setlength{\parskip}{1em}

I like environments that support \textbf{growth}, working along with \textbf{smart}, \textbf{respectful}, and \textbf{open-minded} people who care about their systems, tools, and code.

I love to \textbf{code}! I'm a \textbf{NixOS}, \textbf{terminal}, and \textbf{neovim} user. I often learn new programming languages, frameworks, and technologies. I have worked professionally with Java, Javascript, \textbf{Typescript}, \textbf{Ruby}, and \textbf{Go}, but have also experimented with SML, Python, Haskell, Scala, Rust, Racket, Bash, Clojure, IO, and many others. I designed \href{https://github.com/juanibiapina/marco}{my own toy language}, a \href{https://github.com/basherpm/basher}{package manager for Bash}, and an \href{https://github.com/juanibiapina/sub}{organizer for command line scripts} that I use as part of my \href{https://github.com/juanibiapina/dotfiles}{12 years old dotfiles}. I try to keep at least one active coding side project. You can find more details about personal projects, passions, and open source on my \href{https://github.com/juanibiapina/}{Github Profile} and my \href{https://juanibiapina.github.io/}{blog}.

I’m also a \textbf{game}, \textbf{music}, and \textbf{climbing} enthusiast.

\endgroup

%\cvsection{Skills}
%
%% Don't overuse these \cvtag boxes — they're just eye-candies and not essential. If something doesn't fit on a single line, it probably works better as part of an itemized list (probably inlined itemized list), or just as a comma-separated list of strengths.
%
%\cvtag{Hard-working}
%\cvtag{Eye for detail}\\
%\cvtag{Motivator \& Leader}
%
%\divider\smallskip
%
%\cvtag{C++}
%\cvtag{Embedded Systems}\\
%\cvtag{Statistical Analysis}

\cvsection{Languages}

\cvskill{English}{5}
\cvskill{Portuguese}{5}
\cvskill{German}{2} %% Supports X.5 values.

%% Yeah I didn't spend too much time making all the
%% spacing consistent... sorry. Use \smallskip, \medskip,
%% \bigskip, \vspace etc to make adjustments.
\medskip

\cvsection{Education}

\cvevent{B.Sc.\ in Computer Science}{UFPI - Universidade Federal do Piauí}{Jan 2004 -- Dec 2007}{}

%\cvsection{Referees}
%
%% \cvref{name}{email}{mailing address}
%\cvref{Prof.\ Alpha Beta}{Institute}{a.beta@university.edu}
%{Address Line 1\\Address line 2}
%
%\divider
%
%\cvref{Prof.\ Gamma Delta}{Institute}{g.delta@university.edu}
%{Address Line 1\\Address line 2}


\end{paracol}


\end{document}
